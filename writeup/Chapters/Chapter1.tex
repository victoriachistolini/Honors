% Chapter 1

\chapter{Introduction} % Main chapter title

\label{Chapter1} % For referencing the chapter elsewhere, use \ref{Chapter1} 

%----------------------------------------------------------------------------------------

% Define some commands to keep the formatting separated from the content 
\newcommand{\keyword}[1]{\textbf{#1}}
\newcommand{\tabhead}[1]{\textbf{#1}}
\newcommand{\code}[1]{\texttt{#1}}
\newcommand{\file}[1]{\texttt{\bfseries#1}}
\newcommand{\option}[1]{\texttt{\itshape#1}}

%----------------------------------------------------------------------------------------

% where does main INTENT of the PROJECT GO? OR is that for the abstract????

\section{What is Species Distribution Modeling?}
According to the Center for Disease Control (CDC) in 2015, 95\% of reported Lyme Disease cases came from only 14 states; of these 14, 12 were on the east coast and all 6 of the New England States were represented. Since 1996 the annual number of confirmed cases of Lyme Disease per year has increased by over 10,000 additional cases in 2016 [1]. Given the growing magnitude of the problem presented by ticks and Lyme Disease infection, there has been a deep interest to understand where Lyme Disease carrying ticks are located and how we can most effectively reduce human contact. \newline

\noindent Beginning in 1995, the Maine Medical Center Research Institute (MMCRI) began a project to create detailed records of the locations of discovered infected ticks. Doctors were encouraged to have their patients bring in any ticks that they found on their bodies, in their homes or on their pets, for free testing to determine if the tick were carrying Lyme Disease [2]. Data about the locations of these tick sightings were recorded as part of the study until 2014. \newline

\noindent Species Distribution Modeling (SDM) is a term used to categorize a whole class of models developed for the purpose of understanding the patterns and relationships of an observed species and its environment [3]. Often, the purpose of these models is to predict the range of a species based on where the species has been recorded during surveillance. Another application of SDM models is to predict whether or not a species could be found in a certain location based on the environmental variables of the location. It is with this latter focus that we will pursue Species Distribution models throughout this work. The focus of our research is to develop models to forecast the likelihood of a human-tick encounter in the State of Maine.  \newline

