% Chapter 2

\chapter{Exploratory Analysis} % Main chapter title

\label{Chapter2} % For referencing the chapter elsewhere, use \ref{Chapter1} 

%----------------------------------------------------------------------------------------

% Define some commands to keep the formatting separated from the content 
%\newcommand{\keyword}[1]{\textbf{#1}}
%\newcommand{\tabhead}[1]{\textbf{#1}}
%\newcommand{\code}[1]{\texttt{#1}}
%\newcommand{\file}[1]{\texttt{\bfseries#1}}
%\newcommand{\option}[1]{\texttt{\itshape#1}}

%----------------------------------------------------------------------------------------



\section{Temporal Stationarity Experiments }

The initial experiment in exploratory analysis was to determine how each of the environmental parameters that we are potentially going to be using in the MaxEnt models are related to one another. In order to do this we carry out a correlation analysis to understand how each pairwise set of parameters varies with one another. \newline

\noindent We broke the experiment into four seasons to see if there might be different highly correlated covariates depending on the season. Then we looked at the entire year a more general picture. Below is a section of code used to run the experiment. The dataTools library contains all of the main functions used throughout the code snippet. Here we define 

\begin{figure} [!ht]
\centerline{\psfig{figure=Figures/meta,width=7in}}
\caption{Correlation Analysis between parameters for four seasons, fall (A), summer (B), winter (C), Spring (D) and all year (E).}
\label{fig6}
\end{figure}


\begin{lstlisting} [language=R]

library(dataTools)

window = c(-3,3)
params = c("first_vegtyp", "max_airtemp", "mean_airtemp", "mean_relhum", 
                  "mean_sncvr", "mean_sndep", "mean_trnstr", "mean_uwind", 
                  "mean_vegcvr", "mean_vwind", "mean_wilt", "min_airtemp", 
                  "sum_precip")
           
           
fall_days = c(273, 280, 287, 294, 301, 308, 315, 322, 329, 336)
data = create_multi_week_set(fall_days, window, params)
corr_matrx(data,length(params),"fall.csv")

data_matrx = read.csv("fall.csv")
process_corr_matrix(data_matrx,params)

\end{lstlisting}

\section{Poisson Point Process Models for Predictor Selection and Evaluation}

\subsection{ Building and Interpreting Point Process Models}

\noindent My first look at the Poisson Point Process Models for explaining the distribution of points across the state focused on Day 150 over the years of our experiment 2006-2013. I tried two different window sizes: +/- 3 days and +/- 7 days. Both window periods have lead me to similar conclusions, however the 7 day window had twice as many points so I will discuss my findings for this window.\newline

\noindent Figure 2.2 shows the points observations for the tick sightings on this day of the year for the experiment time period plotted over the state of Maine. We see that there is a good geographic distribution along the highly populated areas of the state. \newline

\noindent We have found three different models that show all highly significant predictors for this day in time:

\begin{equation}
\lambda(i) = e^{-432.54 + 1.37*airtemp^*(i) + 0.49*meanHumidity(i) + 0.038*meanVegcvr(i) - 5.10*uwind(i) + 2.69*vwind(i) + 1.0*sumPrecip(i) }
\end{equation}

\noindent We say $airtemp^* $because you can include here mean, max or min airtemp and come to similar results.


\begin{equation}
\lambda(i) = e^{-483.93 + 1.54*meanAirtemp(i) + 0.58*meanHumidity(i) + 4.46*trnstr(i) - 5*uwind(i) + 1.35*vwind(i) + 0.81*sumPrecip(i) }
\end{equation}
\newline


\section{Predictor Selection Process}

 \begin{longtable}{ |p{3cm}||p{3cm}|p{3cm}|p{3cm}|  }
 \caption{Long table caption.\label{long}}\\

 \hline
 \multicolumn{4}{| c |}{Selected Models}\\
 \hline
 Day of Year & Window Size & Number of Observations & Predictors\\
 \hline
 19   & 20    &12 &   meanAirtemp, vwind, sndepth, sumPrecip \\
 \hline
 19 &   20  & 12   & meanAirtemp, v4 \\
 \hline
 50 & 20 & 11 &  meanAirtemp,  vwind, sncvr\\
 \hline
 50 & 20 & 11 &  meanAirtemp,  vwind, v4\\
 \hline
 78  & 7 & 117 &  meanAirtemp, meanVegcvr, vwind, sumPrecip, sncvr, v4\\
 \hline
 109 &  3   & 118 & meanAirtemp, uwind, meanVegcvr, vwind, minAirtemp, sumPrecip, v4 \\
 \hline
 139 & 3  & 233  & meanAirtemp, uwind, meanVegcvr, vwind, wilt, sumPrecip \\
  \hline
 139 & 3  & 233  & meanAirtemp, meanVegcvr, vwind, sncvr \\
  \hline
 139 & 3  & 233  & meanAirtemp, uwind, meanVegcvr, vwind, wilt, sumPrecip, v4 \\
 \hline
 139 & 3  & 233  & meanAirtemp, meanVegcvr, vwind, sndepth, v4 \\
  \hline
 139 & 3  & 233  & meanAirtemp, meanVegcvr, vwind, sndepth \\
 \hline
 150 & 7  & 445 & meanAirtemp, meanHumidity, meanVegcvr, uwind, vwind, sumPrecip\\
 \hline

 \endfirsthead
 \hline
 \multicolumn{4}{| c |}{Selected Models Continued }\\
 \hline
  Day of Year & Window Size & Number of Observations & Predictors \\
  \hline
  \endhead


 \hline
 150 & 7  & 445 & minAirtemp, meanVegcvr, uwind, vwind, sumPrecip, v4\\
 \hline
150 & 7  & 445 & meanAirtemp, meanHumidity, uwind, sumPrecip, v4\\
 \hline
 150 & 7  & 445 & minAirtemp, meanVegcvr, uwind, vwind, sumPrecip, v4\\
  \hline
 170 & 7  & 262 & meanAirtemp, meanHumidity, uwind, vwind, sumPrecip, v4\\
 \hline
 170 & 7  & 262 & minAirtemp, meanHumidity, uwind, sumPrecip, v4\\
 \hline
 170 & 7  & 262 & minAirtemp, maxAirtemp, meanHumidity, uwind, vwind, sumPrecip, v4\\
 \hline
  170 & 7  & 262 & minAirtemp, maxAirtemp, meanHumidity, uwind, trnstr, sumPrecip, v4\\
 \hline
   170 & 7  & 262 & minAirtemp, maxAirtemp, meanHumidity, uwind, trnstr, sumPrecip, v4\\
 \hline
   170 & 7  & 262 & minAirtemp, maxAirtemp, meanHumidity, uwind, vwind, wilt, sumPrecip, v4\\
 \hline
    200 & 10  & 61 & wilt, sumPrecip, v4\\
 \hline
     231 & 13  & 13 & meanAirtemp\\
 \hline
      231 & 13  & 13 & trnstr\\
 \hline
      231 & 13  & 13 & wilt\\
 \hline
       262 & 10  & 62 & meanAirtemp, uwind, vwind, sumPrecip\\
 \hline
        292 & 3  & 553 & meanAirtemp, uwind, meanVegcvr, v4\\
 \hline
         292 & 3  & 553 & maxAirtemp, uwind, wilt, v4\\
 \hline
          292 & 3  & 553 & maxAirtemp, uwind, trnstr, v4)\\
 \hline
          323 & 3  & 250 & meanAirtemp, trnstr, uwind, vwind, v4)\\
 \hline
           323 & 3  & 250 & meanAirtemp, wilt, uwind, vwind, v4)\\
 \hline
            323 & 3  & 250 & meanAirtemp, meanVegcvr, uwind, vwind, v4)\\
 \hline
             353 & 10  & 34 & meanHumidity, uwind, meanVegcvr, v4)\\
 \hline
\end{longtable}


