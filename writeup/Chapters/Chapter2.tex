% Chapter 2

\chapter{Exploratory Analysis} % Main chapter title

\label{Chapter2} % For referencing the chapter elsewhere, use \ref{Chapter1} 

%----------------------------------------------------------------------------------------

% Define some commands to keep the formatting separated from the content 
%\newcommand{\keyword}[1]{\textbf{#1}}
%\newcommand{\tabhead}[1]{\textbf{#1}}
%\newcommand{\code}[1]{\texttt{#1}}
%\newcommand{\file}[1]{\texttt{\bfseries#1}}
%\newcommand{\option}[1]{\texttt{\itshape#1}}

%----------------------------------------------------------------------------------------

\section{Analyzing Spacial Patterns}

\subsection{What is a point pattern}
Using point process methods: quantify the process (random mechanism) whose outcome is a point pattern [12]. \newline

We want to answer questions about how to points have been distributed in space, are they random, does their distribution depend on a covariate? \newline

We treat each point as a random variable, thus the point itself has its own distribution; follow a Poisson distribution if the probability of a point being in any region is random, ie there is spacial independence. \newline

\subsection{Null Model and Modified Point Processes}
The null model for the point process is complete spacial randomness (CRS); thus the points come from a Poisson distribution and are independent of each other and have homogeneity, also we assume that points are mapped without omission, and that the process is stationary (without autocorrelation) \newline

Inhomogeneous point process: modified Poisson point process such that the average density of points is spatially varying. Assumptions are: ..., small world model \newline

\subsection{Controlling for Processes between points and within points}
Intensity can be quantified by a function of spacial location in an inhomogeneous point process. In order to judge if data is clustered we need to eliminate the possibility that there is spacial variation in intensity, 
and spacial variation in intensity would need to be accounted for in a clustering analysis. \newline


Estimated intensity at or around a point can be use to weight the point and give it greater importance in a dataset. \newline

If covariates are categorical it is necessary to create a tessellated surface from which to measure intensity. The surface can be tessellated by means of the covariate. Can use a chi-squared point counting method to determine if the goodness of fit to data within quadrants. \newline

There can also be inter-point interaction: a stochastic departure, or trend between points. We can model this using the K-function to try and separate correlation between points to covariate explained trends in the data.  
For example, perhaps the points repel each other because of competition for nutrition, or other nuanced interaction between points in an ecological sense. \newline

CURRENT PLAN: \newline

* determine which covariates are correlated with each other and should probably not be included in the model building COV-experiment \newline
COV experiment details:\newline
1 - currently looked at each week from may-sept with background of +/-3 days \newline
2 -  found the average values of each of the parameters \newline
3 - determined pairwise covariance \newline

Design other COV experiments to get full coverage of entire year \newline

Use Poisson point to determine statistical sig of params in model can help further eliminate params\newline

Start work with maxent using 'optimal param set': \newline
begin with AUC / background point experiments \newline
tick data \newline
\pagebreak


\section{Exploratory Data Analysis }

The initial experiment in exploratory analysis was to determine how each of the environmental parameters that we are potentially going to be using in the MaxEnt models are related to one another. In order to do this we carry out a correlation analysis to understand how each pairwise set of parameters varies with one another. \newline

\noindent We broke the experiment into four seasons to see if there might be different highly correlated covariates depending on the season. Then we looked at the entire year a more general picture. Below is a section of code used to run the experiment. The dataTools library contains all of the main functions used throughout the code snippet. Here we define 

\begin{figure} [t]
\centerline{\psfig{figure=Figures/meta,width=7in}}
\caption{Correlation Analysis between parameters for four seasons, fall (A), summer (B), winter (C), Spring (D) and all year (E).}
\label{fig6}
\end{figure}


\begin{lstlisting} [language=R]

library(dataTools)

window = c(-3,3)
params = c("first_vegtyp", "max_airtemp", "mean_airtemp", "mean_relhum", 
                  "mean_sncvr", "mean_sndep", "mean_trnstr", "mean_uwind", 
                  "mean_vegcvr", "mean_vwind", "mean_wilt", "min_airtemp", 
                  "sum_precip")
           
           
fall_days = c(273, 280, 287, 294, 301, 308, 315, 322, 329, 336)
data = create_multi_week_set(fall_days, window, params)
corr_matrx(data,length(params),"fall.csv")

data_matrx = read.csv("fall.csv")
process_corr_matrix(data_matrx,params)

\end{lstlisting}

\section{Poisson Point Process Models for Predictor Selection and Evaluation}

\subsection{ Building and Interpreting Point Process Models}

\noindent My first look at the Poisson Point Process Models for explaining the distribution of points across the state focused on Day 150 over the years of our experiment 2006-2013. I tried two different window sizes: +/- 3 days and +/- 7 days. Both window periods have lead me to similar conclusions, however the 7 day window had twice as many points so I will discuss my findings for this window.\newline

\noindent Figure 2.2 shows the points observations for the tick sightings on this day of the year for the experiment time period plotted over the state of Maine. We see that there is a good geographic distribution along the highly populated areas of the state. \newline

\noindent We have found three different models that show all highly significant predictors for this day in time:

\begin{equation}
\lambda(i) = e^{-432.54 + 1.37*airtemp^*(i) + 0.49*meanHumidity(i) + 0.038*meanVegcvr(i) - 5.10*uwind(i) + 2.69*vwind(i) + 1.0*sumPrecip(i) }
\end{equation}

\noindent We say $airtemp^* $because you can include here mean, max or min airtemp and come to similar results.


\begin{equation}
\lambda(i) = e^{-483.93 + 1.54*meanAirtemp(i) + 0.58*meanHumidity(i) + 4.46*trnstr(i) - 5*uwind(i) + 1.35*vwind(i) + 0.81*sumPrecip(i) }
\end{equation}
\newline


\begin{figure} [t]
\centerline{\psfig{figure=Figures/pts,width=7in,height=9in}}
\caption{Tick observations in the State of Maine on May 30, 2006-2013.}
\label{fig6}
\end{figure}

\section{Predictor Selection Process}

 \begin{longtable}{ |p{3cm}||p{3cm}|p{3cm}|p{3cm}|  }
 \caption{Long table caption.\label{long}}\\

 \hline
 \multicolumn{4}{| c |}{Selected Models}\\
 \hline
 Day of Year & Window Size & Number of Observations & Predictors\\
 \hline
 19   & 20    &12 &   meanAirtemp, vwind, sndepth, sumPrecip \\
 \hline
 19 &   20  & 12   & meanAirtemp, v4 \\
 \hline
 50 & 20 & 11 &  meanAirtemp,  vwind, sncvr\\
 \hline
 50 & 20 & 11 &  meanAirtemp,  vwind, v4\\
 \hline
 78  & 7 & 117 &  meanAirtemp, meanVegcvr, vwind, sumPrecip, sncvr, v4\\
 \hline
 109 &  3   & 118 & meanAirtemp, uwind, meanVegcvr, vwind, minAirtemp, sumPrecip, v4 \\
 \hline
 139 & 3  & 233  & meanAirtemp, uwind, meanVegcvr, vwind, wilt, sumPrecip \\
  \hline
 139 & 3  & 233  & meanAirtemp, meanVegcvr, vwind, sncvr \\
  \hline
 139 & 3  & 233  & meanAirtemp, uwind, meanVegcvr, vwind, wilt, sumPrecip, v4 \\
 \hline
 139 & 3  & 233  & meanAirtemp, meanVegcvr, vwind, sndepth, v4 \\
  \hline
 139 & 3  & 233  & meanAirtemp, meanVegcvr, vwind, sndepth \\
 \hline
 150 & 7  & 445 & meanAirtemp, meanHumidity, meanVegcvr, uwind, vwind, sumPrecip\\
 \hline

 \endfirsthead
 \hline
 \multicolumn{4}{| c |}{Selected Models Continued }\\
 \hline
  Day of Year & Window Size & Number of Observations & Predictors \\
  \hline
  \endhead
 %\hline 
 %\endfoot
 
 %\hline
 %\multicolumn{4}{| c |}{End of Table}\\
 %\hline\hline
 %\endlastfoot


 \hline
 150 & 7  & 445 & minAirtemp, meanVegcvr, uwind, vwind, sumPrecip, v4\\
 \hline
150 & 7  & 445 & meanAirtemp, meanHumidity, uwind, sumPrecip, v4\\
 \hline
 150 & 7  & 445 & minAirtemp, meanVegcvr, uwind, vwind, sumPrecip, v4\\
  \hline
 170 & 7  & 262 & meanAirtemp, meanHumidity, uwind, vwind, sumPrecip, v4\\
 \hline
 170 & 7  & 262 & minAirtemp, meanHumidity, uwind, sumPrecip, v4\\
 \hline
 170 & 7  & 262 & minAirtemp, maxAirtemp, meanHumidity, uwind, vwind, sumPrecip, v4\\
 \hline
  170 & 7  & 262 & minAirtemp, maxAirtemp, meanHumidity, uwind, trnstr, sumPrecip, v4\\
 \hline
   170 & 7  & 262 & minAirtemp, maxAirtemp, meanHumidity, uwind, trnstr, sumPrecip, v4\\
 \hline
   170 & 7  & 262 & minAirtemp, maxAirtemp, meanHumidity, uwind, vwind, wilt, sumPrecip, v4\\
 \hline
    200 & 10  & 61 & wilt, sumPrecip, v4\\
 \hline
     231 & 13  & 13 & meanAirtemp\\
 \hline
      231 & 13  & 13 & trnstr\\
 \hline
      231 & 13  & 13 & wilt\\
 \hline
       262 & 10  & 62 & meanAirtemp, uwind, vwind, sumPrecip\\
 \hline
        292 & 3  & 553 & meanAirtemp, uwind, meanVegcvr, v4\\
 \hline
         292 & 3  & 553 & maxAirtemp, uwind, wilt, v4\\
 \hline
          292 & 3  & 553 & maxAirtemp, uwind, trnstr, v4)\\
 \hline
          323 & 3  & 250 & meanAirtemp, trnstr, uwind, vwind, v4)\\
 \hline
           323 & 3  & 250 & meanAirtemp, wilt, uwind, vwind, v4)\\
 \hline
            323 & 3  & 250 & meanAirtemp, meanVegcvr, uwind, vwind, v4)\\
 \hline
             353 & 10  & 34 & meanHumidity, uwind, meanVegcvr, v4)\\
 \hline
\end{longtable}


