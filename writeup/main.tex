%%%%%%%%%%%%%%%%%%%%%%%%%%%%%%%%%%%%%%%%%
% Masters/Doctoral Thesis 
% LaTeX Template
% Version 2.5 (27/8/17)
%
% This template was downloaded from:
% http://www.LaTeXTemplates.com
%
% Version 2.x major modifications by:
% Vel (vel@latextemplates.com)
%
% This template is based on a template by:
% Steve Gunn (http://users.ecs.soton.ac.uk/srg/softwaretools/document/templates/)
% Sunil Patel (http://www.sunilpatel.co.uk/thesis-template/)
%
% Template license:
% CC BY-NC-SA 3.0 (http://creativecommons.org/licenses/by-nc-sa/3.0/)
%
%%%%%%%%%%%%%%%%%%%%%%%%%%%%%%%%%%%%%%%%%

%----------------------------------------------------------------------------------------
%	PACKAGES AND OTHER DOCUMENT CONFIGURATIONS
%----------------------------------------------------------------------------------------

\documentclass[
11pt, % The default document font size, options: 10pt, 11pt, 12pt
%oneside, % Two side (alternating margins) for binding by default, uncomment to switch to one side
english, % ngerman for German
singlespacing, % Single line spacing, alternatives: onehalfspacing or doublespacing
%draft, % Uncomment to enable draft mode (no pictures, no links, overfull hboxes indicated)
%nolistspacing, % If the document is onehalfspacing or doublespacing, uncomment this to set spacing in lists to single
%liststotoc, % Uncomment to add the list of figures/tables/etc to the table of contents
%toctotoc, % Uncomment to add the main table of contents to the table of contents
%parskip, % Uncomment to add space between paragraphs
%nohyperref, % Uncomment to not load the hyperref package
headsepline, % Uncomment to get a line under the header
%chapterinoneline, % Uncomment to place the chapter title next to the number on one line
%consistentlayout, % Uncomment to change the layout of the declaration, abstract and acknowledgements pages to match the default layout
]{MastersDoctoralThesis} % The class file specifying the document structure
\usepackage[utf8]{inputenc} % Required for inputting international characters
\usepackage[T1]{fontenc} % Output font encoding for international characters
\usepackage{filecontents}
\usepackage{mathpazo} % Use the Palatino font by default
\usepackage{listings}
\usepackage{epsfig}

\usepackage[utf8]{inputenc}
\usepackage{longtable}

\usepackage[backend=bibtex,style=authoryear,natbib=true]{biblatex} % Use the bibtex backend with the authoryear citation style (which resembles APA)

\addbibresource{e.bib} % The filename of the bibliography

\usepackage[autostyle=true]{csquotes} % Required to generate language-dependent quotes in the bibliography




%----------------------------------------------------------------------------------------
%	MARGIN SETTINGS
%----------------------------------------------------------------------------------------

\geometry{
	paper=a4paper, % Change to letterpaper for US letter
	inner=2.5cm, % Inner margin
	outer=3.8cm, % Outer margin
	bindingoffset=.5cm, % Binding offset
	top=1.5cm, % Top margin
	bottom=1.5cm, % Bottom margin
	%showframe, % Uncomment to show how the type block is set on the page
}

%----------------------------------------------------------------------------------------
%	THESIS INFORMATION
%----------------------------------------------------------------------------------------

\thesistitle{Species Distribution Model Development and Validation} % Your thesis title, this is used in the title and abstract, print it elsewhere with \ttitle
\supervisor{Dr. Stephanie \textsc{Taylor}} % Your supervisor's name, this is used in the title page, print it elsewhere with \supname
\examiner{} % Your examiner's name, this is not currently used anywhere in the template, print it elsewhere with \examname
\degree{Computer Science} % Your degree name, this is used in the title page and abstract, print it elsewhere with \degreename
\author{Victoria \textsc{Chistolini}} % Your name, this is used in the title page and abstract, print it elsewhere with \authorname
\addresses{} % Your address, this is not currently used anywhere in the template, print it elsewhere with \addressname

\subject{Biological Sciences} % Your subject area, this is not currently used anywhere in the template, print it elsewhere with \subjectname
\keywords{} % Keywords for your thesis, this is not currently used anywhere in the template, print it elsewhere with \keywordnames
\university{\href{https://www.colby.edu/}{Colby College}} % Your university's name and URL, this is used in the title page and abstract, print it elsewhere with \univname
\department{\href{https://www.bigelow.org/}{Bigelow Laboratory for Ocean Sciences}} % Your department's name and URL, this is used in the title page and abstract, print it elsewhere with \deptname
\group{\href{https://www.bigelow.org/science/lab/computational-science/}{Record Lab}} % Your research group's name and URL, this is used in the title page, print it elsewhere with \groupname
\faculty{\href{http://faculty.university.com}{Dr. Nick Record}} % Your faculty's name and URL, this is used in the title page and abstract, print it elsewhere with \facname

\AtBeginDocument{
\hypersetup{pdftitle=\ttitle} % Set the PDF's title to your title
\hypersetup{pdfauthor=\authorname} % Set the PDF's author to your name
\hypersetup{pdfkeywords=\keywordnames} % Set the PDF's keywords to your keywords
}

\begin{document}

\frontmatter % Use roman page numbering style (i, ii, iii, iv...) for the pre-content pages

\pagestyle{plain} % Default to the plain heading style until the thesis style is called for the body content

%----------------------------------------------------------------------------------------
%	TITLE PAGE
%----------------------------------------------------------------------------------------

\begin{titlepage}
\begin{center}

\vspace*{.06\textheight}
{\scshape\LARGE \univname\par}\vspace{1.5cm} % University name
\textsc{\Large Honors Thesis}\\[0.5cm] % Thesis type

\HRule \\[0.4cm] % Horizontal line
{\huge \bfseries \ttitle\par}\vspace{0.4cm} % Thesis title
\HRule \\[1.5cm] % Horizontal line
 
\begin{minipage}[t]{0.4\textwidth}
\begin{flushleft} \large
\emph{Author:}\\
\href{https://github.com/victoriachistolini}{\authorname} % Author name - remove the \href bracket to remove the link
\end{flushleft}
\end{minipage}
\begin{minipage}[t]{0.4\textwidth}
\begin{flushright} \large
\emph{Supervisor:} \\
\href{https://www.colby.edu/directory/profile/stephanie.taylor/}{\supname} % Supervisor name - remove the \href bracket to remove the link  
\end{flushright}
\end{minipage}\\[3cm]
 
\vfill

\large \textit{A thesis submitted in fulfillment of the requirements\\ for Honors in the degree of \degreename}\\[0.3cm] % University requirement text
\textit{in the}\\[0.4cm]
\groupname\\\deptname\\[2cm] % Research group name and department name
 
\vfill

{\large \today}\\[4cm] % Date
%\includegraphics{Logo} % University/department logo - uncomment to place it
 
\vfill
\end{center}
\end{titlepage}

%----------------------------------------------------------------------------------------
%	ABSTRACT PAGE
%----------------------------------------------------------------------------------------

\begin{abstract}
\addchaptertocentry{\abstractname} % Add the abstract to the table of contents

\noindent In the Northeastern United States, ticks bites and subsequent Lyme Disease infection are becoming a growing problem [1]. Data collected by the Maine Medical Center Research Institute from 1995 - 2013 from citizen reports of ticks, presents the unique opportunity to study patterns of human - tick encounters in the state of Maine [2]. The field of species distribution modeling seeks to develop mathematical and computational models to characterize the distribution of a species based on it environment, and often uses the Maximum Entropy model to accomplish this goal [3]. Tick activity is highly dependent on the time of year and the tick life cycle, thus it is reasonable to expect that a single model will not be able to capture the time-of-year specific drivers of activity. \newline

\noindent Thus we seek to develop an ensemble of Maximum Entropy models that are tuned to forecasting tick encounter likelihood at different times of the year. Ensembles of models can effectively be created by altering the combination of predictor variables in the model [14]. We will use the Poisson point process model as a statistical framework for evaluating predictor combinations at different times of the year. Using the best performing parameter set for each month of the year, we will build Maximum Entropy models on training data and evaluate model fit of training data and model performance on testing data using a modified AUC statistic [12]. 

\end{abstract}

%----------------------------------------------------------------------------------------
%	ACKNOWLEDGEMENTS
%----------------------------------------------------------------------------------------

\begin{acknowledgements}
\addchaptertocentry{\acknowledgementname} % Add the acknowledgements to the table of contents
I would like to thank Stephanie Taylor for her continued guidance, support and mentorship throughout my work on this project. Nick Record, thank you for welcoming me into your lab, introducing me to your research and helping me to learn more about your incredible field of computational - mathematical modeling. A big thanks to Ben Tupper for always welcoming my questions and helping guide me through the massive code base that he has built. Manny Gimond, thank you for introducing me to spatial point pattern analysis and for sharing your course resources and mentorship on the accompanying R programming. 

\end{acknowledgements}

%----------------------------------------------------------------------------------------
%	LIST OF CONTENTS/FIGURES/TABLES PAGES
%----------------------------------------------------------------------------------------

\tableofcontents % Prints the main table of contents

%\listoffigures % Prints the list of figures

%\listoftables % Prints the list of tables


%----------------------------------------------------------------------------------------
%	THESIS CONTENT - CHAPTERS
%----------------------------------------------------------------------------------------

\mainmatter % Begin numeric (1,2,3...) page numbering

\pagestyle{thesis} % Return the page headers back to the "thesis" style

% Include the chapters of the thesis as separate files from the Chapters folder
% Uncomment the lines as you write the chapters

% Chapter 1

\chapter{Introduction} % Main chapter title

\label{Chapter1} % For referencing the chapter elsewhere, use \ref{Chapter1} 

%----------------------------------------------------------------------------------------

% Define some commands to keep the formatting separated from the content 
\newcommand{\keyword}[1]{\textbf{#1}}
\newcommand{\tabhead}[1]{\textbf{#1}}
\newcommand{\code}[1]{\texttt{#1}}
\newcommand{\file}[1]{\texttt{\bfseries#1}}
\newcommand{\option}[1]{\texttt{\itshape#1}}

%----------------------------------------------------------------------------------------

% where does main INTENT of the PROJECT GO? OR is that for the abstract????

\section{What is Species Distribution Modeling}
According to the Center for Disease Control (CDC) in 2015, 95\% of reported Lyme Disease cases came from only 14 states; of these 14, 12 were on the east coast and all 6 of the New England States were represented. Since 1996 the annual number of confirmed cases of Lyme Disease per year has increased by over 10,000 additional cases in 2016 [1]. Given the growing magnitude of the problem presented by ticks and Lyme Disease infection, there has been a deep interest to understand where Lyme Disease carrying ticks are located and how we can most effectively reduce human contact. \newline

\noindent Beginning in 1995, the Maine Medical Center Research Institute (MMCRI) began a project to create detailed records of the locations of discovered infected ticks. Doctors were encouraged to have their patients bring in any ticks that they found on their bodies, in their homes or on their pets, for free testing to determine if the tick were carrying Lyme Disease [2]. Data about the locations of these tick sightings were recorded as part of the study until 2014. \newline

\noindent Species Distribution Modeling (SDM) is a term used to categorize a whole class of models developed for the purpose of understanding the patterns and relationships of an observed species and its environment [3]. Often, the purpose of these models is to predict the range of a species based on where the species has been recorded during surveillance. Another application of SDM models is to predict wether or not a species could be found in a certain location based on the environmental variables of the location. It is with this latter focus that we will pursue Species Distribution models throughout this work. The focus of our research is to develop models to forecast the likelihood of a human-tick encounter in the State of Maine.  \newline

\subsection{Data used in Species Distribution Models}
% mention more detail about data, like predictors (ie raster images each pixel represents a value), each sighting entry has its lat/lon coordinates 
\noindent The database of tick encounters developed by MMCRI houses an enormous amount of information about the distribution of locations where ticks have been found in the state of Maine, however in its raw form, the data is missing a crucial metric: information about where ticks are not found. \newline

\noindent Classical modeling techniques use a set of predictor variables to classify events under certain conditions as likely or unlikely to happen. A logistic regression model, for example could take predictor variables about patients' heart rates and temperatures to determine the likelihood that the patient has the flu. In order to make a good prediction about the patient, the model needs heart rates and temperatures from patients who are healthy and from those who are sick in order to be able to differentiate sick from healthy metrics. In machine learning terminology, models like logistic regression are part of a class of models called supervised learning algorithms, because in order to discern patterns these models need examples of each category to be classified. \newline

\noindent Unfortunately, our tick dataset only provides information about locations where ticks are present and no information about the locations where ticks are absent, which motivates the need to use a different class of models. Unsupervised learning models are able to classify data that has not been labeled, by which group it belongs to, through using the patterns inherent in the data itself to distinguish and predict classification groups. However, in order to use unsupervised algorithms it is necessary that your data contain many examples from each category that you are trying to predict. Thus since our data contain only presence information and no absence data, we are unable build models for our data using unsupervised algorithm as well. \newline

\noindent Clearly when selecting an appropriate model for your data, it is essential to be certain that the type and quality of your data fits what is necessary and expected by the model building procedure to produce respectable results. If the model makes assumptions about your data, which do not hold, then the interpretability and usability of the model will be greatly impaired. In the literature on Species Distribution Modeling, the maximum entropy model is most commonly used, because the assumptions it makes about the data are well aligned to our purpose of modeling, thus we focus our research efforts on creating high preforming models using the maximum entropy model. \newline

%\subsection{Data Quality and Selection Bias} % is this good to have if so where????
%\noindent We need to address selection bias, spacial-auto correlation, correlation with roads, sampling effort is correlated with population density. \newline

%----------------------------------------------------------------------------------------
\section{The Maximum Entropy Model}

To determine the appropriateness of any modeling approach we first start by summarizing what information we know to begin with. We know a sample of locations (latitude and longitude coordinates), where ticks have been found, called presence points. Assuming the presence points are well collected and representative of the locations and environmental conditions where ticks are likely to be found by humans, then we can use this information to estimate and predict locations with high likelihood of encounter. \newline

\noindent We have no information about unsuitable locations, where encounter risk is low, however, we can use our study region, the state of Maine as a constraint on our distribution. By taking a random sample of points from our study region, we create a representation of the diverse habitat conditions in our region, called background points. While we have a high degree of certainty that the presence points represent spaces where ticks should be found, we have no opinion at all about whether or not ticks should be found in the background locations. The theory of minimizing cross entropy ( which can be proved analogous to maximizing entropy)  says that given the information that we do have we can actually find a unique distribution that is optimized to the information that we know and does not penalize us for what we don't know, by selecting the distribution with minimal cross entropy [10].\newline


\noindent More technically put, what we begin with are two probability density functions, $p(x)$ the distribution of tick from our presence and background points and $q(x)$, the probably density function of the tick encounter likelihood conditional on weather conditions and geographic constraints. The process of finding maximum entropy, then is  accomplished by minimize the cross entropy function:  \newline

\begin{equation}
\int q(x)log(\frac{q(x)}{p(x)}) dx
\end{equation}


\noindent  During the minimization process predictor variables undergo transformations that help to maximize the entropy of the solution; this is how the model is fit [6]. The covariates may be transformed into terms of the type : linear, quadratic, product (representing interaction), hinge, threshold or categorical variables in the final model. The effectiveness of the transformations are assessed through cross validation process, using L1 regulation to avoid overfitting [6].  \newline


\subsection{Maximum Entropy Example}

\noindent When we began the modeling process, we had our dataset of tick observations, which highlighted the set of locations where ticks are know to be able to survive, we also start with another subset of point locations in the state of Maine, called the background points, which represent the landscape that we are trying to model. Perhaps the background points dataset consists of two environmental features, elevation and vegetation type look like this: 

\begin{center}
 \begin{tabular}{||c c c c||} 
 \hline
 x & Elevation & Vegetation Type & p(x) \\ [0.5ex] 
 \hline\hline
 1 & H &  1& 0.1  \\ 
 \hline
  2 & H & 3 & 0.1\\
 \hline
   3 & H & 2 & 0.1\\
 \hline
 4 & M & 2 & 0.1  \\
 \hline
  5 & L & 4 & 0.1 \\
 \hline
   6 & L & 3 & 0.1\\
 \hline
  7 & M & 5 & 0.1 \\
 \hline
   8 & M & 4 & 0.1\\
 \hline
 9 & L & 3 & 0.1 \\
 \hline
   10 & M & 3 & 0.1\\
 \hline
\end{tabular}
\end{center}


\noindent We know nothing more than this about the background locations, so we distribute probability to the subset of locations uniformly. That is if we were to go back to each of these locations then the chances of us getting a tick is equally in each location; so if we have $n$ observations in our set, then the probability of getting a tick at any of the locations would be $\frac{1}{n}$. This is a very naive estimate that each location is equiprobable this is our best first guess at the distribution and is called the prior distribution. By preforming a spacial analysis we may be able to find patterns between the locations of our tick sightings in the observations dataset, for example that 90\% of the locations were below a certain elevation. Based on this insight, which we call a constraint, we can recalculate the probabilities predicted for each location in our background point dataset. \newline

\begin{center}
 \begin{tabular}{||c c c c||} 
 \hline
 x & Elevation & Vegetation Type & p(x) \\ [0.5ex] 
 \hline\hline
 1 & H &  1& 1/30 \\ 
 \hline
  2 & H & 3 &  1/30\\
 \hline
   3 & H & 2 &  1/30\\
 \hline
 4 & M & 2 & 9/70  \\
 \hline
  5 & L & 4 & 9/70  \\
 \hline
   6 & L & 3 & 9/70 \\
 \hline
  7 & M & 5 & 9/70 \\
 \hline
   8 & M & 4 &  9/70 \\
 \hline
 9 & L & 3 & 9/70 \\
 \hline
   10 & M & 3 & 9/70\\
 \hline
\end{tabular}
\end{center}

\noindent Next we define another constraint on the model; we find that 82\% of the tick sightings are happening at vegetation types 4 and 5. We can incorporate this new constraint and reweigh the model. This time there is not as clear cut a way to reweigh the probabilities. We know 18\% of the probability should be represented in the set $S_1$= \{ 1, 2, 3, 4, 6, 9,10\} and 80\% should be represented in the set $S_2$ = \{5,7,8 \} but how should the probabilities be determined for each element in the set? As mentioned previously, based on the principle of maximum entropy, we want to select the distribution that is most uniform the distribution by minimizing the entropy function in equation 1.2. This equation is difficult to solve analytically, thus numerical methods will be used for the optimization procedure. This toy example is based off of work done in [11]. \newline

\noindent In building the maxent model distribution, we have used constraints in order to add information to our model. There are several methods for figuring out how to derive the constraint rules. One of these processes is similar to the construction of a decision tree where at each new additional leaf we want to choose the feature that maximizes the information gain. We use the environmental features to develop the constraints which are the expected values of the features.\newline

\subsection{Current MaxEnt model}

\noindent The algorithm that we will be using to create our maximum entropy model is called MaxEnt and is supported by the R programming language through the dismo package; the main algorithm is written in Java, by S.J. Phillips et al. We have currently developed several different MaxEnt models using environmental covariates that best fit the expert knowledge of the parameters which affect tick survival. The core set of environmental covariates come from North American Mesoscale Forecast System (NAM), in the form of raster images of a certain resolution. Each pixel of the image represents a projected latitude, longitude coordinate and the value stored at the pixel is the value of the covariate at that location. For example if the raster image represented elevation, then the value at each pixel, would be the meters above sea level at that location. \newline

\noindent The core parameter set includes: minimum air temperature, maximum air temperature, mean air temperature, mean percent vegetation cover, mean relative humidity, mean snow cover, mean snow depth, mean transpiration rate, mean u-direction wind speed, mean v-direction wind speed, mean wilt, sum of precipitation. Some models include additional parameters for previous-year's minimum winter temperatures.  \newline

\noindent Once a MaxEnt model is fit using the covariates mentioned above, the output model is of the form:
\begin{equation}
q(x) = \frac{e^{\lambda* f(x)}}{Z}
\end{equation}

\noindent where $\lambda $ is a set of weights on the features and Z is a scaling constant that makes sure the probability distribution $q(x)$ sums to 1 [12]. A new model is fit for each day of the year using all of the data in the database available for the particular day of the year. Since often there is not enough data for a single day of the year, data is taken from a window of time around the model day. The size for a suitable window depends on the time of year, with larger windows being needed during time of the year with fewer tick observations.\newline

\noindent Since we have created several MaxEnt models our goal is preform exploratory analysis and look at potential sources of error and model refinement through the creation of new models with time-of-year specific parameter combinations to boost overall system performance. Since a main goal of our work is to develop a rigorous understanding of how covariates influence model performance, when need a statistically based framework to assess significance of parameter effect. The MaxEnt model does not allow us to preform hypothesis tests on coefficient significance for the output expression, thus we must use other methodology from spacial-temporal analysis to develop inference about covariates. 


%----------------------------------------------------------------------------------------
\section{Spacial-Temporal Exploratory Analysis} 

Exploratory data analysis is a crucial step in the model development process, it should occur before any model is attempted, it is here that the statistician is able to get to know the data. The key component of exploratory analysis is data visualization [14]. We want to know what values our response and predictor variables can take on, what do their distributions look like? Is there missing data? Does the scale of the raw data make sense, or should we transform the data, perhaps using a $log(x)$ transform or discretizing data which only takes on certain integer values. There is no standard procedure for exploratory analysis, however the major goals of the process is the answer questions in regard to the distribution and quality of the data, as well as how well data would be able to hold up the assumptions necessary for preforming future modeling efforts.\newline

\noindent Aside from the broader questions mentioned above, our exploratory analysis seeks to also answer questions about the Spacial-Temporal aspects of our data. Spacial-Temporal analysis is the use of specialized statistical techniques that are designed to help identify patterns that derive from the spacial and temporal dimensions of the data [4].  In the sections below we address the questions we seek to answer through spacial-temporal analysis of the data, as well as the methodologies that we will use to accomplish these goals. \newline

\subsection{Spacial Analysis and Predictor Selection}
\noindent A branch of spacial analysis called point pattern analysis can help us understand how the tick observations are distributed spatially across different covariates and provide a statistical framework for hypothesis testing of these covariates. We will be using the poisson point process model for our experiments. A point process is a way of mechanistically thinking about point distributions, that assumes points are generated according to some process, this process may be random, or it may be defined in terms of external factors [13]. \newline

\noindent One important kind of point process is that generated by a completely random mechanism, which we will call a point process with complete spacial randomness (CSR). CSR point processes have two special properties, (1) they are homogeneous, meaning that the intensity of points does not depend on the location within the experiment window, and (2) the points are independent of one another; meaning that if we know a certain number of points are in one region we cannot predict the number of points in another region [13]. An inhomogeneous point process (IHP), is a point process such that intensity of points depends on spacial location [13]. We will want to develop a test that can determine if our tick data in a CSR pattern, the null hypothesis or an IHP pattern, the alternative hypothesis. \newline 

\noindent In order to develop the statistical test mentioned above, we need to make some observations about point patterns. If we assume that the study space of the points is well defined and that an observation could have occurred anywhere in the study space, and that the set of observations is a complete enumeration of all observations, then it can be shown that the points follow a poisson distribution. Informally, this can be seen by imagining the study space divided into infinitesimally small regions, such that most regions have no points, then by the law of the frequency of rare events, the points follow a poisson distribution [13].  \newline

\noindent Since we suspect that environmental factors are the major mechanisms driving the human-tick observation pattern, then the human-tick observation pattern is an IHP type pattern. We want to test the hypothesis, if we control for the covariate mechanisms, does the IHP become a CSR. Figure 1.1 provides an graphic explanation of this process: in A the IHP from day 150 and years 2006-2013 clearly is clustered in space. It is important to note that the right boundary, added in blue in B is the Atlantic Ocean and is actually not part of the study region for the purpose of calculations. In B the boundaries of a fictitious covariate are added, one could think of these boundaries as temperature gradient lines. One can see that each of the sections in between the gradient lines has a CSR pattern such that points are not spatially varying and the points are independent. Although this example has been fabricated it is a good illustration of what we seek to do, I make not be a simple as adding a single covariate to achieve CSR, one might need to add several terms, including interaction and indicator terms as well.\newline
 
 
\noindent The fitted poison point process model is of the form:

\begin{equation}
\lambda(x) = e^{\alpha +\beta(x)}
\end{equation}

\noindent We can conduct hypothesis tests on each of the $\beta_i$ coefficients to discern the effectiveness of the parameters in the model. \newline


\begin{figure} [t]
\centerline{\psfig{figure=Figures/CSR_exp.pdf,width=4in,height=3in}}
\caption{An example of Complete Spacial Randomness derived by controlling for covariates. }
\label{fig6}
\end{figure}

\noindent We must be careful about some of the assumptions that we made, and how they relate to our interpretation of the model. The assumption of having the points be a complete enumeration is clearly not met. This causes the model to penalize all areas that do not have points as it assess the patterns. Although this assumption might be too destructive if we were trying to create a mapping of all of the possible locations for ticks, this is not our goal. We are trying to map human-tick encounter risk and so even though the places were no observations are seen are actually likely possible habitats, it is ok that they are penalized because these places with few to no observations have much lower populations and thus it is much less likely for people to encounter tick there in the first place.\newline


\subsection{Temporal Analysis}
% do predictor sets change over time
Ideally we are trying to create a forecasting model for each day of the year. The need for models that are season or even month-day specific is based off of the fact that the weather conditions can vary greatly throughout the year, as with the behavioral patterns of the humans who are going to encounter ticks. Therefore, when designing our experiments of predictor selection, it is important to repeat the experiments to capture the yearly variability. One way to do this is to run a predictor selection experiment for a day mid-month for each month in the year. \newline

% window experiments
\noindent Another temporal component of analysis concerns the amount of point data that the dataset has at different times of the year. Ideally we will be looking to make a model for each day of the year to use for forecast prediction. However, due to the fact that one is less likely to encounter a tick in winter months, we have fewer observations for this time frame. With fewer observations to build the model, the forecast predictions are less accurate. \newline 

\noindent Thus, we must preform an experiment to understand the impact of the amount of data used to build the model and the model's subsequent accuracy. We seek to understand if adding more data from the observations of surrounding months to the model when there are few observations will positively impact accuracy. We also seek to determine a threshold about the number of observations necessary to achieve a certain error tolerance and a bound on the window size based on day of the year. \newline

% stationarity of the data set
\noindent Another area of interest is the stationarity of the data. There is evidence that the range of suitable tick habitats continues to expand as the effects of global climate change become more dramatic [5]. At this point in our primary analysis we have not explicitly done any on how, if present at all the lack of stationarity of data is present in our database. We expect that there will be some correlation with the tick distribution, its predictors and time. In order to test this correlation, we can build regression models for each predictor at each location across time to quantify the degree and strength of non-stationarity in the data.  \newline


%----------------------------------------------------------------------------------------


\section{Model Assessment }  % FIX include stuff about FAUC 

\subsection{Assessment of Performance }

% draw a confusion matrix here....
Assessment of classification algorithms is an essential part of determining their utility as valid predictive models. Classically testing the strength of a classification algorithm involves building the model with a subset of the dataset and keeping another subset for use in testing the model called the training dataset. Then the model is run with the training data and an assessment is made about how well the model has done. The construction of a confusion matrix helps to communicate how well the model did in classifying the training data, by tabulating the number of correctly classified results as well as false positives and false negatives. 

\begin{equation} \bordermatrix{~ & Observed \hspace{0.2cm} Presence & Observed \hspace{0.2cm} Absence \cr
                  Predicted\hspace{0.2cm}  Presence & a & b \cr
                  Predicted \hspace{0.2cm} Absence & c & d \cr} \end{equation}
                  
\noindent The data in the confusion matrix can then be visualized in a plot called the ROC curve. \newline

\subsection{The AUC statistic for model assessment }

The ROC curve plots the classifier's sensitivity versus 1-specificity, where sensitivity is easily calculated from the confusion matrix as  $\frac{a}{a+c}$ and 1- specificity is $\frac{b}{b+d}$ [7]. Since the results of the MaxEnt model are a series of probabilities given environmental conditions of encountering a tick at a particular location, then in order to create a confusion matrix, we would need to decide on some arbitrary threshold at which we decide that a probability is high enough to be considered a presence of a tick. \newline

\noindent Since the decision of a threshold is arbitrary then ROC is formed from finding the confusion matrix for each threshold, which will give a new sensitivity and 1-specificity value to plot for each matrix. The summary statistic used to characterize the ROC curve is the area under the ROC curve or AUC, which evaluates the strength of the classifier by the characteristics of the ROC curve. An AUC statistic of 0.5 represents a random classifier, and scores above 0.5 represent a better than random model [7].    \newline

\noindent Given the type of data we are using, presence only data, and the fact that we are generating a forecast, the traditional methods of model assessment fall short. Firstly, since the training dataset would be only a series of presence observations, so there would be 0 observations correctly classified for no encounter, cell d,  which would make it impossible to calculate summary statistics, used to calculated the ROC curve. Given these two complicating factors, we will pursue other methods of assessment which employ modifications on the classical techniques to be functional for a presence only dataset.  \newline

\noindent Since we do not have absence data in our dataset and thus our training dataset is devoid of this essential information, then a more accurate substitute is using is using the proportion of area predicted present instead of 1-specificity [7]. With the simple modification under way we can proceed to interpret the AUC in a similar fashion as our interpretations of the traditionally defined AUC.  \newline


\subsection{The F-AUC statistic for assessment of future predictive power }

The F-AUC or forecast AUC, is another validation technique used to assess the strength a performance of models. I model is built with a given predictor set, using only a subset of the data in the database. For example years 2006-2010, then the model is tested using data from 2011-2013, using the 2006-2010 model to predict the outcomes for the 2011-2013 data. Using this metric we can get a more accurate estimate of the true predictive power of the model. 

\subsection{Limitations of AUC Interpretability}
\noindent Another limiting factor that about AUC that we have previously mentioned is that it represents the classification ability of the model independent of a threshold. The threshold determines the minimum probability that will be considered a presence. In our application this threshold is important because we are creating a forecast and the interpretability of a forecast greatly depends on the ability to discriminate events from non-events, thus the there seems to be an implied necessity to define the threshold as a single number when calculating a summary statistic about the skill of the forecast. The statistic True Skill Statistic (TSS):
\begin{equation}
TSS \hspace{0.2cm} = \hspace{0.2cm} sensitivity \hspace{0.2cm} + \hspace{0.2cm} specificity\hspace{0.2cm}  -\hspace{0.2cm} 1 \hspace{0.2cm} =  \hspace{0.2cm} \frac{ad -bc}{(a+c)(b+d)}
\end{equation}
can be used used to quantify the strength of the classifier at a given threshold [9]. The TSS statistic has been shown to have a good behavior and is well correlated with the AUC statistic [9]. However, in order to calculate the TSS statistic we would need to have access to a complete confusion matrix which we do not have, thus in order to use the TSS, we would need to substitute our proxy metric calculated at a given threshold. \newline



%----------------------------------------------------------------------------------------

\section{Ensemble Models}

Since we are trying to develop models to forecast human-tick encounter for the entire year, it is reasonable to assume that it is unlikely that a single model will be able provide the best predictions for every day of the year. However, it is reasonable to believe that a collection models with different parameter combinations could be specialized to preform better at different times of the year. Evidence has shown that collections of models, called ensemble models provide more robust forecast models, where each of the individual models in the ensemble provides independent and novel information to contribute to a collective consensus [15]. \newline

\noindent An ensemble of models is defined as creating duplicated models with altered initial conditions, boundary conditions, types of model, and parameter combinations [15]. We we create a model there are many sources of uncertainty, we do not know what the true mechanisms that drive human-tick encounters, so we must take educated guesses at which predictors are useful to include. We began by intuitively selecting parameters and used the statistical frame work of the poisson point process to guide predictor selection for different times of the year.  We also preformed exploratory experiments on the window size parameter at different times of the year, perturbing it based on season and subsequent observation density.  \newline

\noindent Despite having created many new models, looking specifically at conditions at certain times of the year to fit the highly variable patterns of observations conditional on time of year, it is possible that our ensemble, as many ensembles are, is underdispersive meaning that the ensemble is not as variable as would be expected given the diversity of its members [16]. \newline

\noindent Variability is desirable in an ensemble because, it ensures robustness and representation of future potential variability [15] Underdispersion occurs because it is very difficult to capture all of the types of uncertainty that exist and thus there is a trade off between exhaustively searching the multi-dimensional space of model uncertainty versus spending effort in creating an ensemble of fewer, perhaps less variable, but individually more skillful members. There is evidence that focusing energy on improving the quality of individual models, produces higher quality results. [15]


\subsection{Future Work with Ensembles}
 
Currently we have done the first step of creating an ensemble by selecting a set of skillful models that have strengths at different times of the year. Currently we do not have enough information to distinguish which models preform better than others conditional on the time of the year. Future work includes more rigorous testing of the ensemble members on future collected data and storing model performance over the long term. With this information we would be able to discern which models are more skill full an properly calibrate the ensemble as described in [16]. 


%----------------------------------------------------------------------------------------

% Chapter 1

\chapter{Approach} % Main chapter title

\label{Chapter2} 

%----------------------------------------------------------------------------------------


\subsection{Data used in Species Distribution Models}
% mention more detail about data, like predictors (ie raster images each pixel represents a value), each sighting entry has its lat/lon coordinates 
\noindent The database of tick encounters developed by MMCRI houses an enormous amount of information about the distribution of locations where ticks have been found in the state of Maine, however in its raw form, the data is missing a crucial metric: information about where ticks are not found. \newline

\noindent Classical modeling techniques use a set of predictor variables to classify events under certain conditions as likely or unlikely to happen. A logistic regression model, for example could take predictor variables about patients' heart rates and temperatures to determine the likelihood that the patient has the flu. In order to make a good prediction about the patient, the model needs heart rates and temperatures from patients who are healthy and from those who are sick in order to be able to differentiate sick from healthy metrics. In machine learning terminology, models like logistic regression are part of a class of models called supervised learning algorithms, because in order to discern patterns these models need examples of each category to be classified. \newline

\noindent Unfortunately, our tick dataset only provides information about locations where ticks are present and no information about the locations where ticks are absent, which motivates the need to use a different class of models. Unsupervised learning models are able to classify data that has not been labeled, by which group it belongs to, through using the patterns inherent in the data itself to distinguish and predict classification groups. However, in order to use unsupervised algorithms it is necessary that the data contain many examples from each category that you are trying to predict. Thus since our data contain only presence information and no absence data, we are unable build models for our data using unsupervised algorithms. \newline

\noindent Clearly when selecting an appropriate model for a dataset, it is essential to be certain that the type and quality of the data fits what is necessary and expected by the model building procedure to produce respectable results. If the model makes assumptions about the data, which do not hold, then the interpretability and usability of the model will be greatly impaired. In the literature on Species Distribution Modeling, the maximum entropy model is most commonly used, because the assumptions it makes about the data are well aligned with our purpose of modeling. Thus, we focus our research efforts on creating high performing models using the maximum entropy model. \newline

%\subsection{Data Quality and Selection Bias} % is this good to have if so where????
%\noindent We need to address selection bias, spacial-auto correlation, correlation with roads, sampling effort is correlated with population density. \newline

%----------------------------------------------------------------------------------------
\section{The Maximum Entropy Model}

To determine the appropriateness of any modeling approach we first start by summarizing what information we know in the beginning. We know a sample of locations (latitude and longitude coordinates), where ticks have been found, called presence points. Assuming the presence points are well collected and representative of the locations and environmental conditions where ticks are likely to be found by humans, then we can use this information to estimate and predict locations with high likelihood of encounter. \newline

\noindent We have no information about unsuitable locations, where encounter risk is low, however, we can use our study region, the state of Maine as a constraint on our distribution. By taking a random sample of points from our study region, we create a representation of the diverse habitat conditions in our region, called \textit{background points}. While we have a high degree of certainty that the \textit{presence points} represent spaces where ticks should be found, we have no opinion at all about whether or not ticks should be found in the background locations. The theory of minimizing cross entropy ( which can be proved analogous to maximizing entropy)  says that given the information that we have, we can actually find a unique distribution that is optimized to the information that we know and does not penalize us for what we don't know, by selecting the distribution with minimal cross entropy [4].\newline


\noindent Put more technically, what we begin with are two probability density functions, $p(x)$ the distribution of ticks from our presence and background points and $q(x)$, the probably density function of the tick encounter likelihood conditional on weather conditions and geographic constraints. The process of finding maximum entropy, then is  accomplished by minimizing the cross entropy function:  \newline

\begin{equation}
\int q(x)log(\frac{q(x)}{p(x)}) dx
\end{equation}


\noindent  During the minimization process, the predictor variables undergo transformations that help to maximize the entropy of the solution; this is how the model is fit [5]. The covariates may be transformed into terms of the type : linear, quadratic, product (representing interaction), hinge, threshold or categorical variables in the final model. The effectiveness of the transformations are assessed through cross validation process, using L1 regulation to avoid overfitting [5].  \newline


\subsection{Maximum Entropy Example}

\noindent When we began the modeling process, we had our dataset of tick observations, which highlighted the set of locations where ticks are known to be able to survive. We also start with another subset of point locations in the state of Maine, called the background points, which represent the landscape that we are trying to model. Perhaps the background points dataset consists of two environmental features, elevation and vegetation type and looks like this: 

\begin{center}
 \begin{tabular}{||c c c c||} 
 \hline
 x & Elevation & Vegetation Type & p(x) \\ [0.5ex] 
 \hline\hline
 1 & H &  1& 0.1  \\ 
 \hline
  2 & H & 3 & 0.1\\
 \hline
   3 & H & 2 & 0.1\\
 \hline
 4 & M & 2 & 0.1  \\
 \hline
  5 & L & 4 & 0.1 \\
 \hline
   6 & L & 3 & 0.1\\
 \hline
  7 & M & 5 & 0.1 \\
 \hline
   8 & M & 4 & 0.1\\
 \hline
 9 & L & 3 & 0.1 \\
 \hline
   10 & M & 3 & 0.1\\
 \hline
\end{tabular}
\end{center}


\noindent We know nothing more than this about the background locations, so we distribute probability to the subset of locations uniformly. That is if we were to go back to each of these locations then the chances of us getting a tick is equal in each location; so if we have $n$ observations in our set, then the probability of getting a tick at any of the locations would be $\frac{1}{n}$. This is a very naive estimate that each location is equiprobable, but this is our best first guess at the distribution and is called the prior distribution. By performing a spacial analysis we may be able to find patterns between the locations of our tick sightings in the observations dataset, for example that 90\% of the locations were below a certain elevation. Based on this insight, which we call a constraint, we can recalculate the probabilities predicted for each location in our background point dataset. \newline

\begin{center}
 \begin{tabular}{||c c c c||} 
 \hline
 x & Elevation & Vegetation Type & p(x) \\ [0.5ex] 
 \hline\hline
 1 & H &  1& 1/30 \\ 
 \hline
  2 & H & 3 &  1/30\\
 \hline
   3 & H & 2 &  1/30\\
 \hline
 4 & M & 2 & 9/70  \\
 \hline
  5 & L & 4 & 9/70  \\
 \hline
   6 & L & 3 & 9/70 \\
 \hline
  7 & M & 5 & 9/70 \\
 \hline
   8 & M & 4 &  9/70 \\
 \hline
 9 & L & 3 & 9/70 \\
 \hline
   10 & M & 3 & 9/70\\
 \hline
\end{tabular}
\end{center}

\noindent Next we define another constraint on the model; we find that 82\% of the tick sightings are happening at vegetation types 4 and 5. We can incorporate this new constraint and reweight the model. This time there is not as clear cut a way to reweight the probabilities. We know 18\% of the probability should be represented in the set $S_1$= \{ 1, 2, 3, 4, 6, 9,10\} and 80\% should be represented in the set $S_2$ = \{5,7,8 \} but how should the probabilities be determined for each element in the set? As mentioned previously, based on the principle of maximum entropy, we want to select the most uniform distribution by minimizing the entropy function in equation 1.2. This equation is difficult to solve analytically, thus numerical methods will be used for the optimization procedure. This toy example is based off of work done in [6]. \newline

\noindent In building the maxent model distribution, we have used constraints in order to add information to our model. There are several methods for figuring out how to derive the constraint rules. One of these processes is similar to the construction of a decision tree where at each new additional leaf we want to choose the feature that maximizes the information gain. We use the environmental features to develop the constraints which are the expected values of the features.\newline

\subsection{Current MaxEnt model}

\noindent The algorithm that we will be using to create our maximum entropy model is called MaxEnt and is supported by the R programming language through the dismo package; the main algorithm is written in Java, by S.J. Phillips et al (2006) [7]. We have currently developed several different MaxEnt models using environmental covariates that best fit the expert knowledge of the parameters that affect tick survival. The core set of environmental covariates came from North American Mesoscale Forecast System (NAM), in the form of raster images of a 12 km resolution. Each pixel of the image represents a projected latitude-longitude coordinate, and the value stored at the pixel is the value of the covariate at that location. For example if the raster image represented elevation, then the value at each pixel would be the meters above sea level at that location. \newline

\noindent The core parameter set includes: minimum air temperature, maximum air temperature, mean air temperature, mean percent vegetation cover, mean relative humidity, mean snow cover, mean snow depth, mean transpiration rate, mean u-direction wind speed (east-west winds), mean v-direction wind speed (north-south winds), mean wilt, sum of precipitation. Some models include additional parameters for previous-year's minimum winter temperatures.  \newline

\noindent Once a MaxEnt model is fit using the covariates mentioned above, the output model is of the form:
\begin{equation}
q(x) = \frac{e^{\lambda* f(x)}}{Z}
\end{equation}

\noindent where $\lambda $ is a set of weights on the features and Z is a scaling constant that makes sure the probability distribution $q(x)$ sums to 1 [7]. A new model is fit for each day of the year using all of the data in the database available for the particular day of the year. Since often there is not enough data for a single day of the year, data is taken from a window of time around the model day. The size for a suitable window depends on the time of year, with larger windows being needed during time of the year with fewer tick observations.\newline

\noindent Our next goal is to perform exploratory analysis and look at potential sources of error and model refinement through the creation of new models with time-of-year specific parameter combinations to boost overall system performance. Since one main goal of our work is to develop a rigorous understanding of how covariates influence model performance, we need a statistically based framework to assess significance of parameter effect. The MaxEnt model does not allow us to perform hypothesis tests on coefficient significance for the output expression, thus we must use other methodology from spatial-temporal analysis to develop inference about covariates. 


%----------------------------------------------------------------------------------------
\section{Spacial-Temporal Exploratory Analysis} 

Exploratory data analysis is a crucial step in the model development process. It should occur before any model is attempted. It is here that the statistician is able to get to know the data. The key component of exploratory analysis is data visualization [8]. Exploratory analysis tells us something about the patterns generated from the data. It is also useful in identifying problematic elements of the data such as missing values or the nature of the scales at which the measurements are made. For example, does the scale of the raw data make sense, or should we transform the data, perhaps using a $log(x)$ transform or discretizing data which only takes on certain integer values. There is no standard procedure for exploratory analysis, however the major goal of the process is to  assess the data quality and to reveal data patterns. The same exploratory analysis tools are also used in examining the assumptions necessary for any parametric hypothesis tests.\newline

\noindent Aside from the broader questions mentioned above, our exploratory analysis seeks to also answer questions about the Spatial-Temporal aspects of our data. Spatial-Temporal analysis is the use of specialized statistical techniques that are designed to help identify patterns that derive from the spatial and temporal dimensions of the data [9].  In the sections below we address the questions we seek to answer through spatial-temporal analysis of the data, as well as the methodologies that we will use to accomplish these goals. \newline

\subsection{Spatial Analysis and Predictor Selection}
\noindent A branch of spatial analysis called \textit{point pattern analysis} can help us understand how the tick observations are distributed spatially across different covariates and provide a statistical framework for hypothesis testing of these covariates. We will be using the \textit{poisson point process} model for our experiments. A point process is a way of mechanistically thinking about point patterns and the processes that generate these patterns. These processes may be random, or they may be defined in terms of external factors [10]. \newline

\noindent One basic point process model is that generated by a completely random mechanism, which we will call a point process with complete spatial randomness (CSR). CSR point processes have two special properties, (1) they are homogeneous, meaning that the intensity of points does not depend on the location within the experiment window, and (2) the points are independent of one another; meaning that the location of one point cannot influence the placement of a nearby point [10]. An inhomogeneous point process (IHP), is a point process such that the intensity of points depends on spatial location [10]. We will want to develop a test that can determine if our tick data follow a CSR process (the null hypothesis) or an IHP pattern ( the alternative hypothesis). \newline 

\noindent In order to develop the statistical test mentioned above, we need to make some observations about point patterns. If we assume that the study space of the points is well defined and that an observation could have occurred anywhere in the study space, and that the set of observations is a complete enumeration of all observations, then it can be shown that the points follow a poisson distribution. Informally, this can be seen by imagining the study space divided into infinitesimally small regions, such that most regions have no points, then by the law of the frequency of rare events, the points follow a poisson distribution [10].  \newline

\noindent Since we suspect that environmental factors are the major mechanisms driving the human-tick observation pattern, then the human-tick observation pattern is an IHP type pattern. We want to test this hypothesis  that human-tick observations follow an IHP, by controlling for the covariate process and then testing for a CSR process. Figure 1.1 provides an graphic explanation of this process: in A the IHP from day 150 and years 2006-2013 clearly is clustered in space. It is important to note that the right boundary, added in blue in B is the Atlantic Ocean and is not part of the study region for the purpose of calculations. In B the boundaries of a fictitious covariate are added, one could think of these boundaries as temperature gradient lines. The point patterns in each section appear to be more in line with a CSR process in that the points appear to be randomly distributed and independent of one another. Although this example has been fabricated it is a good illustration of what we seek to do. However, controlling for a single covariate may not be enough to achieve CSR, one might need to add several terms, including interaction and indicator terms as well.\newline
 
 
\noindent The fitted poison point process model is of the form:

\begin{equation}
\lambda(x) = e^{\alpha +\beta(x)}
\end{equation}

\noindent We can conduct hypothesis tests on each of the $\beta_i$ coefficients to discern the effectiveness of the parameters in the model. \newline


\begin{figure} [t]
\centerline{\psfig{figure=Figures/CSR_exp.pdf,width=4in,height=3in}}
\caption{An example of Complete Spatial Randomness derived by controlling for covariates. }
\label{fig6}
\end{figure}

\noindent We must be careful about some of the assumptions that we made, and how they relate to our interpretation of the model. The assumption of having the points be a complete enumeration is clearly not met. This causes the model to penalize all areas that do not have an observation where a tick incident occurs (due to the event not be recorded). Although this assumption might be too restrictive if we were trying to create a mapping of all of the possible locations for ticks, this is not our goal, we are trying to map human-tick encounter risk. Thus we can argue in favor of penalizing the locations with few to no observations if they have lower populations and thus little chance of an encounter there in the first place\newline


\subsection{Temporal Analysis}
% do predictor sets change over time
Ideally we are trying to create a forecasting model for each day of the year. The need for models that are season or even month-day specific is based on the fact that the weather conditions can vary greatly throughout the year, as with the behavioral patterns of the humans who are going to encounter ticks. Therefore, when designing our experiments of predictor selection, it is important to repeat the experiments to capture the yearly variability. One way to do this is to run a predictor selection experiment for a day mid-month for each month in the year. \newline

% window experiments
\noindent A temporal limitation to the analysis is the limited number of observations at different times of the year. Ideally we will be looking to make a model for each day of the year to use for forecast prediction. However, due to the fact that one is less likely to encounter a tick during the winter months, we have fewer observations for this time frame. With fewer observations to build the model, the forecast predictions are less accurate. \newline 

\noindent Thus, we must perform an experiment to understand the impact the limited amount of data has on the model's subsequent accuracy. We seek to understand if aggregating data from months with few observations will positively impact accuracy. We also seek to determine a threshold about the number of observations necessary to achieve a certain error tolerance and a bound on the window size based on day of the year. \newline

% stationarity of the data set
\noindent Another area of interest is the stationarity of the data. There is evidence that the range of suitable tick habitats continues to expand as the effects of global climate change become more dramatic [11]. At this point in our primary analysis we have not explored stationarity in our dataset. We expect that there will be some correlation with the tick distribution, its predictors and time. In order to test this correlation, we can build regression models for each predictor at each location across time to quantify the degree and strength of non-stationarity in the data.  \newline


%----------------------------------------------------------------------------------------


\section{Model Assessment }  % FIX include stuff about FAUC 

\subsection{Assessment of Performance }

% draw a confusion matrix here....
Assessment of classification algorithms is an essential part of determining their utility as valid predictive models. Classically testing the strength of a classification algorithm involves building the model with a subset of the dataset and keeping another subset for use in testing the model called the training dataset. Then the model is run with the training data and an assessment is made about the model performance. The construction of a confusion matrix helps to communicate how well the model did in classifying the training data by tabulating the number of correctly classified results as well as false positives and false negatives. 

\begin{equation} \bordermatrix{~ & Observed \hspace{0.2cm} Presence & Observed \hspace{0.2cm} Absence \cr
                  Predicted\hspace{0.2cm}  Presence & a & b \cr
                  Predicted \hspace{0.2cm} Absence & c & d \cr} \end{equation}
                  
\noindent The data in the confusion matrix can then be visualized in a plot called the ROC curve. \newline

\subsection{The AUC statistic for model assessment }

The ROC curve plots the classifier's sensitivity versus 1-specificity, where sensitivity is easily calculated from the confusion matrix as  $\frac{a}{a+c}$ and 1- specificity is $\frac{b}{b+d}$ [12]. Since the results of the MaxEnt model are a series of probabilities given environmental conditions of encountering a tick at a particular location, then in order to create a confusion matrix, we would need to decide on some arbitrary threshold at which we decide that a probability is high enough to be considered a presence of a tick. \newline

\noindent Since the decision of a threshold is arbitrary, then an ROC is formed from finding the confusion matrix for each threshold, which will give a new sensitivity and 1-specificity value to plot for each matrix. The summary statistic used to characterize the ROC curve is the area under the ROC curve or AUC, which evaluates the strength of the classifier by the characteristics of the ROC curve. An AUC statistic of 0.5 represents a random classifier, and scores above 0.5 represent a better than random model [12].    \newline

\noindent Since we are using presence only data, the traditional methods of model assessing the ROC curve fall short. The training dataset is a series of presence observations, thus there would be 0 observations correctly classified for no encounter, cell d,  which would make it impossible to calculate 1- specificity. Since we can't calculate 1-specificity, we can proxy it using the proportion of area predicted present [12] [MORE DETAILS].  With the simple modification under way we can proceed to interpret the AUC in a similar fashion as our interpretations of the traditionally defined AUC.  \newline


\subsection{The F-AUC statistic for assessment of future predictive power }

The F-AUC or forecast AUC, is another validation technique used to assess the strength and performance of models. A model is built with a given predictor set, using only a subset of the data in the database, a training dataset, then performance is assessed using the rest of the data, the testing dataset. For example years 2006-2010 of data are used to build the model, then the model is tested using data from 2011-2013, using the 2006-2010 model to predict the outcomes for the 2011-2013 data. Partitioning our data into testing and training sets allows us to more accurately estimate of the true predictive power of the model. 

\subsection{Limitations of AUC Interpretability}
\noindent One limitation of the AUC is that it represents the classification ability of the model independent of a threshold. The threshold determines the minimum probability that will be considered a presence. In our application this threshold is important because we are creating a forecast and the interpretability of a forecast greatly depends on the ability to discriminate events from non-events, thus the there seems to be an implied necessity to define the threshold as a single number when calculating a summary statistic about the skill of the forecast. The statistic True Skill Statistic (TSS):
\begin{equation}
TSS \hspace{0.2cm} = \hspace{0.2cm} sensitivity \hspace{0.2cm} + \hspace{0.2cm} specificity\hspace{0.2cm}  -\hspace{0.2cm} 1 \hspace{0.2cm} =  \hspace{0.2cm} \frac{ad -bc}{(a+c)(b+d)}
\end{equation}
can be used used to quantify the strength of the classifier at a given threshold [13]. The TSS statistic has been shown to have a good behavior and is well correlated with the AUC statistic [13]. However, in order to calculate the TSS statistic we would need to have access to a complete confusion matrix which we do not have. Therefore, in order to use the TSS, we would need to substitute our proxy metric calculated at a given threshold. \newline



%----------------------------------------------------------------------------------------

\section{Ensemble Models}

Since we are trying to develop models to forecast human-tick encounter for the entire year, it is reasonable to assume that it is unlikely that a single model will be able provide the best predictions for every day of the year. However, it is reasonable to believe that a collection of models with different parameter combinations could be specialized to perform better at different times of the year. Evidence has shown that collections of models, called ensemble models, provide more robust forecast models, where each of the individual models in the ensemble provides independent and novel information to contribute to a collective consensus [14]. \newline

\noindent An ensemble of models is defined as creating duplicated models with altered initial conditions, boundary conditions, types of model, and parameter combinations [14]. When we create a model there are many sources of uncertainty, we do not know the true mechanisms that drive human-tick encounters. As a result, we must take educated guesses at which predictors are useful to include. We began by intuitively selecting core parameters and then use the statistical framework of the poisson point process to guide predictor selection for different times of the year.  We also performed exploratory experiments on the window size parameter at different times of the year, perturbing it based on season and subsequent observation density.  \newline

\noindent Despite having created many new models, looking specifically at conditions at certain times of the year to fit the highly variable observation patterns, it is possible that our ensemble is underdispersive. An underdispersive ensemble means that the ensemble is not as variable as would be expected given the diversity of its members [15]. \newline

\noindent Variability is desirable in an ensemble because, it ensures robustness and representation of future potential variability [14] Underdispersion occurs because it is very difficult to capture all of the types of uncertainty that exist. Therefore, is a trade off between exhaustively searching the multi-dimensional space of model uncertainty versus spending effort in creating an ensemble of fewer, perhaps less variable, but individually more skillful members. There is evidence that focusing energy on improving the quality of individual models produces higher quality results. [14]


\subsection{Future Work with Ensembles}
 
Currently we have done the first step of creating an ensemble by selecting a set of skillful models that have strengths at different times of the year. Currently we do not have enough information to distinguish which models perform better than others conditional on the time of the year. Future work includes more rigorous testing of the ensemble members on future collected data and storing model performance over the long term. With this information we would be able to discern which models are more skill full an properly calibrate the ensemble as described in [15]. 


%----------------------------------------------------------------------------------------
 
% Chapter 2

\chapter{Exploratory Analysis and Model Development} % Main chapter title

\label{Chapter2} % For referencing the chapter elsewhere, use \ref{Chapter1} 

%----------------------------------------------------------------------------------------

We will begin a close analysis of the Maine Tick database following the outline of the methodology in the Approach section. We begin by conducting an exploratory analysis on the tick data, covariates and window size model parameter. We want to understand how the tick data and covariates are distributed and how this may change throughout the year. We want to understand the implications of the window size parameter in the model. Specifically, we want to know how the model fit (AUC) is related to different values of window size. \newline

\noindent Then we will use the statistical framework of the Poisson point process model to evaluate how the predictive effectiveness of different covariates changes throughout the year by studying 13 days of the year (the 15th day of each month). The Poisson point process models will guide our predictor selection conditional on time of the year. We will then use the time of year specific covariate combinations to build maximum entropy (MaxEnt) models using training data (data from 2006-2010) with each of these combinations, in an effort to build a forecast ensemble. \newline

\noindent We will evaluate the model fit against the training data using AUC and then evaluate model performance on testing data (data from 2011-2013) using the FAUC. The FAUC will guide a preliminary analysis of the ensemble, but further work will be required in developing rigorous tests for ensemble performance. 

\section{Looking at distributions }

We start by looking at how the raw data of our observations set is distributed. We are looking to create a testing and training dataset that are distributed similarly. Figure 3.1 shows two plots of the number of observations recorded on a given day of the year for the time period 2006-2010 (A) and 2011-2013 (B). Although in (A), we see that there are overall more points per day, the yearly shape of observation records is strikingly similar between these two sets. The first and third quarter of the year have few observations per day, while the second and forth quarters of the year have increasing activity, which spikes mid quarter and then declines for the second half of the quarter. \newline

\begin{figure} [!ht]
\centerline{\psfig{figure=Figures/obs_dist.pdf,width=4in,height=3in}}
\caption{Distribution of tick observations during the year for (A) years 2006-2010 and (B) year 2011-2013. }
\label{fig6}
\end{figure}


\noindent Given the yearly activity cycle of the tick, we need to be cognizant of the number of points being used to create each model. Clearly, there will be a much smaller pool of observations during the first and third quarters. Many days may have zero observations even as we are aggregating data from a span of almost 10 years. On days with so few data points it becomes impossible to create a model. One way around this obstacle of low observations counts is to pull in observations from a window of time around the forecast date. We will call this new parameter the window size of the model. \newline

\section{Window size implications }

\noindent The window size parameter has a lot of uncertainty around it. It is unclear what values this parameter should take on at different times of the year. Further we do not know the impact of increasing the window size parameter on the accuracy and precision of the model. One hypothesis is that it is necessary to keep the window size parameter large enough so that the model has exposure to enough data to create well-informed predictions. Another hypothesis is that if the window size is increased by too much, then the accuracy of the predictions will be weakened due to the presence of data irrelevant to the current stage of tick activity. \newline

\noindent In order to assess the impact of window size on forecast skill, we run a MaxEnt model with a subset of the 13 core parameters as predictors against 7 candidate window sizes: $\pm$ 2, 3, 7, 15, 20, 30, 40 days, for 13 days of the year (approximately the 15th day of each month is tested); days 15, 46, 76, 105, 135, 146, 166, 196, 227, 258, 288, 319, 349 of the year. \newline

\begin{figure} [!ht]
\centerline{\psfig{figure=Figures/wsize.png,width=4in,height=3in}}
\caption{AUC score by day of year for the 2006-2013 time period of observations. Each line represents a different window size from the 7 candidate sizes. }
\label{fig6}
\end{figure}

\noindent The general trend of figure 3.2 independent of window size is the inverse of figure 3.1. Highest AUC's are seen during the first and third quarters of the year, while steep crashes in AUC scores are experienced during the second and forth quarters. It is important to note that as a rule, the fewer points we have in our model, the higher the AUC will be because there is a lower chance of being wrong as described in section 2.4.3 and demonstrated in figure 2.2. Thus figure 3.2 shows that models during low activity are better fit to the training data, than those models developed during high activity periods, however it is unclear if high AUC is correlated with good long term forecast performance or if it is an indicator of an overfit model. \newline

\noindent The question of window size is particularly relevant during the low activity of the first and third quarters of the year. We want to increase the window size to get more data and more information about trends in tick encounters during the time of year, but we don't want the window size to be too big and provide irrelevant information. For the first 25\% of the year it seems that the accuracy is about equal for all window sizes, and thus it is likely useful to include a larger window size so that the model has access to more information at fit time. By around day 70 though, this is not entirely true, windows that are too large such as the sizes 30 and 40 produce much weaker AUC scores. \newline

\noindent The window size of 15, however, continue to perform well high, so it would be advisable to use window sizes up to 15 during this time of the year. Even maintaining a window size as large as 7 into the beginning and at the end of second quarter seems justified based on continued high performance of the 7 day window during these transitional times. For the third quarter a window of size 15 seems to have the best balance of performance and inclusion of data. For the second and fourth quarters, windows sizes or 1 - 2 days are sufficient for high performance because there is plenty of data amassed even for small windows during this time. By the end of the fourth quarter, however, as winter begins to take hold, window sizes of 15 or 20 are best. \newline

\begin{figure} [!ht]
\centerline{\psfig{figure=Figures/num_obs.pdf,width=3in,height=4in}}
\caption{AUC score by number of observations for the 2006-2013 time period of observations. Each dot comes from one of the 13 days of the year mentioned above, with one of the 7 window sizes. }
\label{fig6}
\end{figure}

\noindent Another metric that we can use to assess the window size parameter is by the number of points contained within a given window size. Figure 3.3 shows a decreasing trend of AUC values produced by having too large of a window size. Up to around 200 points the models seem to produce high quality results with AUC scores hovering around 0.9. However, once we go past 500 points, AUC scores start to tank, indicating that window sizes generating such high point values are probably too big to provide valuable information.  \newline



\section{Understanding predictor variables}

Before we try to fit models to our observation set, it is necessary to understand what the distributions of our predictor variables look like. This will allow us to identify is there might be some kinds of transformations to the covariates that might be beneficial to our model. In Figure 3.4, we have created histograms of all of the predictor variables for day 150 of 2007. Day 150 is around the height of the first peak in tick activity. We can see that variables like mean air temperature and wind have distributions that appear continuous in nature. However, transpiration rate, wilt and vegetation cover appear to have discrete distributions of values. \newline

\begin{figure} [!ht]
\centerline{\psfig{figure=Figures/2007_day_150.jpg,width=4in,height=4in}}
\caption{Histograms of the distributions of core predictor  variables for day 150 of 2007. }
\label{fig6}
\end{figure}

\noindent The vegetation cover variable is categorical, with 20 different categories, based on the International Geosphere-Biosphere Programme (IGBP) land cover classification system. According to figure 3.4, the most popular vegetation type for tick observations is 11, permanent wetlands, as well as 12-15 which represent cropland and mix vegetation types of forest, shrub and grassland. A more through examination about how vegetation cover of where ticks are found changes throughout the year, reveled another important vegetation type in the second and fourth quarters of the year to be 5, mixed forests. Since a categorical variable with 20 categories does not seem reasonable since so few categories are actually represented, we create a binary indicator variable, called v4 that represents if a tick is found in vegetation type 11-15 or not. 


\section{Poisson point process models for predictor selection and evaluation}

Using the insights that we gained from the exploratory analysis of our dataset we begin with the first step of the ensemble creation process. In section 1.5, we outlined the process for creating an ensemble which involves perturbing initial conditions (in our case window size and study space boundaries), using different model types or creating novel parameter combinations. We will focus on creating novel parameter combinations that are specialized to certain times of the year and using the statistical frame work of the poisson point process model to assess the significance of each of the parameters in the new models. \newline

\noindent We build poisson point process models for 13 days of the year (approximately the 15th day of each month is tested); days 15, 46, 76, 105, 135, 146, 166, 196, 227, 258, 288, 319, and 349 of the year. We derive the window size parameter by trying to minimize the window size while maintaining a good coverage of observation data over the study space. If increasing window size does not seem to add more coverage and 50 or more points are included, then we do not expand the window size. These judgements are made qualitatively by looking at point spread across the map of Maine. Sometimes, however, we increase the window size as much as up to $\pm$ 20 days, a size which evidence from the previous exploratory window experiments reveals is about as large as possible without sacrificing performance over the long term, and we still only have a few points. Despite having little information, we build a model with what we have.  \newline

\begin{figure} [!ht]
\centerline{\psfig{figure=Figures/pseudo_code.png,width=4in,height=2in}}
\caption{Pseudo code for the predictor selection algorithm. }
\label{fig6}
\end{figure}
\subsection{ Predictor reduction algorithm}

For each of the 13 days of the year mentioned above, download the tick observation data from the time span of 2006-2013, with the window size chosen by hand using the method described above. Data about the window sizes selected for each of the experiments run is shown in table 3.1. We also extract data for each of the core 13 covariates from the raster images at the pixels where each observation in the day and window range being studied occurred. \newline

\noindent We begin parameter selection by generating a correlation matrix of all of the 13 covariates. Trying to create a poisson point process model with covariates that are highly correlated impairs the performance of the model. Once we have information about the highly correlated predictors we are able to create subsets of the parameters that do not contain any of the highly correlated predictors. Then we build a poisson point process model using the command ppm() from the spatstat package in r. Once we have a model built, we prune unnecessary predictors using a greedy, stepwise algorithm, which removes the predictor with the lowest z-score (an indication of its statistical significance) until all predictors are significant at the 5\% level.  \newline

\subsection{ Results from predictor selection experiments}

Figure 3.6 shows the summarized results of the predictor selection experiments, while table 2.1 shows detailed descriptions of the number of observations and number of points used for each experiment. In figure 3.6, highlights many striking patterns throughout the year. We see that during the winter and early spring, the predictors mean air temperature and v-direction wind appear consistently up to day 150 in the year. \newline

\noindent Based on figure 3.1, day 150 occurs approximately at the first peak in tick activity, which coincides with the onset of late spring / early summer. The predictors of highest importance during this time period are the relative humidity, the temperature range (minimum air temperature and maximum air temperature), the type of vegetation and the amount of precipitation. Late summer and fall show a renewed importance of mean air temperature, u-direction wind and vegetation type.  \newline

\begin{figure} [!ht]
\centerline{\psfig{figure=Figures/summary.png,width=4in,height=4in}}
\caption{Histograms of the distributions of core predictor  variables for day 150 of 2007. }
\label{fig6}
\end{figure}


 \begin{longtable}{ |p{3cm}||p{3cm}|p{3cm}|p{3cm}|  }
 \caption{Parameter combinations  and meta data of ensemble members\label{long}}\\

 \hline
 \multicolumn{4}{| c |}{Selected Models}\\
 \hline
 Day of Year & Window Size & Number of Observations & Predictors\\
 \hline
 19   & 20    &12 &   meanAirtemp, vwind, sndepth, sumPrecip \\
 \hline
 19 &   20  & 12   & meanAirtemp, v4 \\
 \hline
 50 & 20 & 11 &  meanAirtemp,  vwind, sncvr\\
 \hline
 50 & 20 & 11 &  meanAirtemp,  vwind, v4\\
 \hline
 78  & 7 & 117 &  meanAirtemp, meanVegcvr, vwind, sumPrecip, sncvr, v4\\
 \hline
 109 &  3   & 118 & meanAirtemp, uwind, meanVegcvr, vwind, minAirtemp, sumPrecip, v4 \\
 \hline
 139 & 3  & 233  & meanAirtemp, uwind, meanVegcvr, vwind, wilt, sumPrecip \\
  \hline
 139 & 3  & 233  & meanAirtemp, meanVegcvr, vwind, sncvr \\
  \hline
 139 & 3  & 233  & meanAirtemp, uwind, meanVegcvr, vwind, wilt, sumPrecip, v4 \\
 \hline
 139 & 3  & 233  & meanAirtemp, meanVegcvr, vwind, sndepth, v4 \\
  \hline
 139 & 3  & 233  & meanAirtemp, meanVegcvr, vwind, sndepth \\
 \hline
 150 & 7  & 445 & meanAirtemp, meanHumidity, meanVegcvr, uwind, vwind, sumPrecip\\
 \hline

 \endfirsthead
 \hline
 \multicolumn{4}{| c |}{Selected Models Continued }\\
 \hline
  Day of Year & Window Size & Number of Observations & Predictors \\
  \hline
  \endhead


 \hline
 150 & 7  & 445 & minAirtemp, meanVegcvr, uwind, vwind, sumPrecip, v4\\
 \hline
150 & 7  & 445 & meanAirtemp, meanHumidity, uwind, sumPrecip, v4\\
 \hline
 150 & 7  & 445 & minAirtemp, meanVegcvr, uwind, vwind, sumPrecip, v4\\
  \hline
 170 & 7  & 262 & meanAirtemp, meanHumidity, uwind, vwind, sumPrecip, v4\\
 \hline
 170 & 7  & 262 & minAirtemp, meanHumidity, uwind, sumPrecip, v4\\
 \hline
 170 & 7  & 262 & minAirtemp, maxAirtemp, meanHumidity, uwind, vwind, sumPrecip, v4\\
 \hline
  170 & 7  & 262 & minAirtemp, maxAirtemp, meanHumidity, uwind, trnstr, sumPrecip, v4\\
 \hline
   170 & 7  & 262 & minAirtemp, maxAirtemp, meanHumidity, uwind, trnstr, sumPrecip, v4\\
 \hline
   170 & 7  & 262 & minAirtemp, maxAirtemp, meanHumidity, uwind, vwind, wilt, sumPrecip, v4\\
 \hline
    200 & 10  & 61 & wilt, sumPrecip, v4\\
 \hline
     231 & 13  & 13 & meanAirtemp\\
 \hline
      231 & 13  & 13 & trnstr\\
 \hline
      231 & 13  & 13 & wilt\\
 \hline
       262 & 10  & 62 & meanAirtemp, uwind, vwind, sumPrecip\\
 \hline
        292 & 3  & 553 & meanAirtemp, uwind, meanVegcvr, v4\\
 \hline
         292 & 3  & 553 & maxAirtemp, uwind, wilt, v4\\
 \hline
          292 & 3  & 553 & maxAirtemp, uwind, trnstr, v4)\\
 \hline
          323 & 3  & 250 & meanAirtemp, trnstr, uwind, vwind, v4)\\
 \hline
           323 & 3  & 250 & meanAirtemp, wilt, uwind, vwind, v4)\\
 \hline
            323 & 3  & 250 & meanAirtemp, meanVegcvr, uwind, vwind, v4)\\
 \hline
             353 & 10  & 34 & meanHumidity, uwind, meanVegcvr, v4)\\
 \hline
\end{longtable}

\section{Creating MaxEnt models}

Having determined combinations of predictor variables that are specialized for different times of the year, we can now build maximum entropy models with these new predictor combinations. We create a new maxent model for each of the 13 studied days of the year, using the window sizes listed in table 3.1, for each of the 32 predictor sets, resulting in 416 maxent models. We calculate the AUC statistic of each model to indicate how well the models fit the data that they were trained on. \newline

\noindent Looking back at figure 3.1 we see the two plots of the number of tick observations on given days of the year. We see that the year trace in figure 3.1 (A)  from 2006-2010 and 2011-2013 in (B) have the same trajectory and thus represent a good split of the data. We use the time span from 2006-2010 to build the 416 maxent models and will use the time span 2011-2013 in section 3.6 to evaluate model performance on testing data that it has never seen before. \newline

\subsection{ MaxEnt model fit results}

One way to easily assess model performance is to evaluate how well the model fits the data that it was trained on. We can think of this step in model evaluation like interpreting a correlation coefficient. The correlation coefficient tells us how well a linear model fits the trends in the data, however it does not indicate anything about how well the model will preform in the task of accurately classifying future data. The AUC statistic provides the same function for MaxEnt models, acting as an indication of goodness of fit. \newline

\begin{figure} [!ht]
\centerline{\psfig{figure=Figures/model_aucs.png,width=6in,height=4in}}
\caption{AUC results from running the 32 different covariate combinations. }
\label{fig6}
\end{figure}


\noindent Figure 3.7 shows the AUC for 32 predictor combinations MaxEnt models fitted at each of the 13 time points in the year. In the winter months, there is a large degree of variability between the models, which tightens during the early spring. As we approach mid-way through the second quarter mode fit tanks and there is enlarged variability again. In the third quarter, the models regain some of their quality performance around mid summer, which remains stable although highly varied until fall what the variability decreases and peaks quickly, followed by a quick crash and slow regain of fit into the new year. 


\section{Evaluating MaxEnt models}

Now that we have built and assessed the goodness of fit of all of our 416 MaxEnt models, we want to utilize the testing dataset that we saved from years 2011-2013 to evaluate the model's performance on data that is has not yet seen. Figures 3.8 and 3.9 show the F-AUC scores of each of the models at the 13 time points of the experiment. The models in figure 3.8 so poor performance up through the first half of the first quarter at around day 60. From a peak at day 60, the models begin a gradual decline in performance up through the beginning of the fourth quarter. During this time period the variability between models is rather low. During the fourth quarter we see increased performance and greater between model variability. \newline

\noindent Although it intuitively make seem that it is good for the ensemble to have low variability, since it may seem like the models are providing a consistent answer, this is not the case, it is ok to have variability between models and in fact this is a desired quality of an ensembles. When ensemble have too little variability between models it is a sign of ucnderdispersion and indicates that the ensemble needs calibration [15]. So it is a good sign that we see variability in our ensemble and not one model dominating the other models, a contrast to what we saw in figure 3.7. \newline


\begin{figure} [!ht]
\centerline{\psfig{figure=Figures/group1_Fauc.png,width=6in,height=4in}}
\caption{F-AUC results from a subset of the 32 predictor combinations that have high performance year round. }
\label{fig6}
\end{figure}

\noindent Figure 3.9 shows the second group of models from the 32 original sets. This group is characterized by two steep crashes in performance in the third and fourth quarters of the year. At the beginning of the year, these models do much better than those in figure 3.8, along with their high performance there is a healthy variability and no single model is dominating the field. High volatility is seen in the third and beginning of the fourth quarter, and it is unclear what is causing this, since the steep descents do not seem to line up with periods of highest activity unlike in figure 3.7.  The models recover performance by the end of the fourth quarter. \newline


\begin{figure} [!ht]
\centerline{\psfig{figure=Figures/group2_Fauc.png,width=6in,height=4in}}
\caption{F-AUC results from a subset of the 32 predictor combinations that have two steep dips in performance throughout the round. }
\label{fig6}
\end{figure}

\noindent  If we look at the patterns of the covariate combinations that make up the models in figure 3.8 and figure 3.9, we can see some interesting patterns. In Figure 3.10, we can see that the models in figure 3.8, who have a fairly consistent trend year long, have a high emphasis on the vegetation cover and vegetation type parameters as well as both u and v direction wind and mean air temperature. \newline

\noindent However, we see in figure 3.11, that the models that have more yearly volatility emphasize parameters such as the range of air temperature ( minimum and maximum air temperature) as well as sum of precipitation, u-direction wind, but most surprisingly is the almost complete absence of the mean vegetation cover predictor. \newline

\begin{figure} [t]
\centerline{\psfig{figure=Figures/group1.png,width=4in,height=2in}}
\caption{Summary of predictor combinations from figure 2.8. }
\label{fig6}
\end{figure}

\begin{figure} [t]
\centerline{\psfig{figure=Figures/group2.png,width=4in,height=2in}}
\caption{Summary of predictor combinations from figure 2.9. }
\label{fig6}
\end{figure}

  
\section{Future directions}

Now that we have identified a set of candidate models and have evidence that these models provide a decent amount of variability and quality performance, it is necessary to collect more data about their performance over the log term. By keeping track of how well the models do on future data, we will then be able to build up a better understanding of long term performance of the models and different points in the year so that we can come up with a posterior weighted averaging of the models based on the knowledge that we have gained. This will help calibrate the ensemble and create a further robust and effective system. \newline


%\include{Chapters/Chapter4} 
%\include{Chapters/Chapter5} 

%----------------------------------------------------------------------------------------
%	THESIS CONTENT - APPENDICES
%----------------------------------------------------------------------------------------

%\appendix % Cue to tell LaTeX that the following "chapters" are Appendices

% Include the appendices of the thesis as separate files from the Appendices folder
% Uncomment the lines as you write the Appendices

%% Appendix A

\chapter{Code for Experiments} % Main appendix title

\label{AppendixA} % For referencing this appendix elsewhere, use \ref{AppendixA}

\section{Experiment 1}

The code for this experiment has been written in R

%\include{Appendices/AppendixB}
%\include{Appendices/AppendixC}

%----------------------------------------------------------------------------------------
%	BIBLIOGRAPHY
%----------------------------------------------------------------------------------------


% Chapter 1

\chapter{Sources} % Main chapter title
[1] CDC https://www.cdc.gov/lyme/stats/index.html \newline

\noindent [2] Record, Nick \newline

\noindent [3] Elith, Jane Leathwick, John R,
"Species Distribution Models: Ecological Explanation and Prediction Across Space and Time" \newline
https://doi.org/10.1146/annurev.ecolsys.110308.120159\newline

\noindent [4] Manuel Gimond, "Introduction to GIS", Colby College \newline

\noindent [5] J. S. Gray, H. Dautel, A. Estrada-Pe, O. Kahl, and E. Lindgren, 
"Effects of Climate Change on Ticks and Tick-Borne Diseases in Europe,"
Interdisciplinary Perspectives on Infectious Diseases, vol. 2009, Article ID 593232,
12 pages, 2009. doi:10.1155/2009/593232\newline

\noindent [6] Elith, Jane, et al. "A Statistical Explanation of MaxEnt for Ecologists." Diversity and Distributions, 
vol. 17, no. 1, 2010, pp. 43?57., doi:10.1111/j.1472-4642.2010.00725.x.\newline

\noindent [7] Peterson, A. Townsend, et al.
"Rethinking Receiver Operating Characteristic Analysis Applications in Ecological Niche Modeling."
Ecological Modelling, vol. 213, no. 1, 2008, pp. 63?72., doi:10.1016/j.ecolmodel.2007.11.008. \newline

\noindent [8] Peterson, A. Townsend, et al. 
"Transferability and Model Evaluation in Ecological Niche Modeling: a Comparison of GARP and Maxent."
Ecography, vol. 30, no. 4, 2007, pp. 550?560.,
doi:10.1111/j.2007.0906-7590.05102.x.\newline

\noindent [9] Allouche, Omri, et al. 
"Assessing the Accuracy of Species Distribution Models: Prevalence, Kappa and the True Skill Statistic (TSS)." 
Journal of Applied Ecology, vol. 43, no. 6, Dec. 2006, pp. 1223?1232., 
doi:10.1111/j.1365-2664.2006.01214.x. \newline

\noindent [10]  Shore, J., and R. Johnson. 
"Axiomatic Derivation of the Principle of Maximum Entropy and the Principle of Minimum Cross-Entropy." 
IEEE Transactions on Information Theory, vol. 26, no. 1, 1980, pp. 26?37.,
doi:10.1109/tit.1980.1056144. \newline

\noindent [11] Berger, Adam. \newline 
https://www.cs.cmu.edu/afs/cs/user/aberger/www/html/tutorial/tutorial.html\newline

\noindent [12] Phillips, Steven J., et al. 
"Maximum Entropy Modeling of Species Geographic Distributions."
Ecological Modelling, vol. 190, no. 3-4, 2006, pp. 231?259.,\newline
doi:10.1016/j.ecolmodel.2005.03.026. \newline

\noindent[13] Spatial Point Patterns Methodology and Applications with R
Adrian Baddeley-Ege Rubak-Rolf Turner - Crc Press - 2016 \newline

\noindent [14] https://mgimond.github.io/ES218/Week01.html \newline

\noindent [15] Thuiller, Wilfried, et al. "BIOMOD - a Platform for Ensemble Forecasting 
of Species Distributions." Ecography, vol. 32, no. 3, 2009, pp. 369?373.,
doi:10.1111/j.1600-0587.2008.05742.x. \newline
 
 \noindent [16] Raftery, Adrian E., et al. 
"Using Bayesian Model Averaging to Calibrate Forecast Ensembles." 2003,
 doi:10.21236/ada459828.
 
 
 
 

%----------------------------------------------------------------------------------------

\end{document}  